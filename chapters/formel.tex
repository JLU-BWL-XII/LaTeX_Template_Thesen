%!TEX root = ../master.tex
\chapter{Fromeln}
Unter Verwendung der schnell Schreibweise mittels eines einzelnen \emph{\$}-Zeichens kann relativ simpel eine Formel im Text verwendet werden.
$ c(\tau) =\frac{1}{|C|(|C|-1)} \sum_{j \in C\backslash \{i\}}1_{d_{ij}>\tau}$\\\\
Unter Verwendung der schnell Schreibweise mittels zwei \emph{\$\$}-Zeichen wird die Formel in einem eigenen Absatz angezeigt:

$$ c(\tau) =\frac{1}{|C|(|C|-1)} \sum_{j \in C\backslash \{i\}}1_{d_{ij}>\tau}$$

Unter Verwendung der Umgebung \emph{align} wird die Formel automatisch Nummeriert.
\begin{align}
    c(\tau) =\frac{1}{|C|(|C|-1)} \sum_{j \in C\backslash \{i\}}1_{d_{ij}>\tau}
\end{align}

$$X_i^2$$