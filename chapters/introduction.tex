%!TEX root = ../master.tex
\chapter{Introduction} \label{chap:introduction}

Wissenschaftliche Arbeiten weisen standardisierte Strukturen auf, von denen man im Ausnahmefall abweichen kann. Eine Standardstruktur ist die folgende:
\begin{itemize}
  \item Abstract
  \item Introduction [Einleitung]
  \item Related Work [Grundlagen und Verwandte Arbeiten]
  \item Method/Experimental Design/Implementation
  \item Results
  \item Discussion
  \item Limitations and Future Research [Limitationen und Ausblick]
  \item Conclusion [Fazit/Schlussfolgerung]
  \item References [Literaturverzeichnis]
  \item Appendix [Anhang]
\end{itemize}

In Bachelor-und Masterthesen können Sie für jeden dieser Abschnitte den Befehl \emph{chapter} wählen. In Seminararbeiten verwenden Sie stattdessen bitte \emph{section}.

Die typische Struktur der Einleitung ist:
\begin{enumerate}
  \item Problemstellung und Motivation
  \item Stand der Forschung, darauf aufbauend Forschungslücke und Forschungsfrage(n) herausarbeiten
  \item Ziel der Arbeit und eigener methodischer Ansatz zur Beantwortung der Forschungsfrage(n): Absatz beginnt meist mit: The goal of this thesis/work/manuscript is
  \item Ergebnisse der Arbeit [optional]
  \item Erwarteter wissenschaftlicher (und praktischer) Beitrag [=Contribution]
  \item Manchmal folgt noch eine Gliederung [ausformuliert]
\end{enumerate}
