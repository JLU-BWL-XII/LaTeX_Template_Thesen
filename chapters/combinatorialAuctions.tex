\chapter{Combinatorial Auctions}
 \label{chap:combinatorialAuctions}

Die ist der Ausschnitt eines Beispielkapitels.

\emph{Combinatorial auctions} (CAs) are a part of electronic
market design. Research in electronic market design joins two
disciplines: economics and computer science. Economical research
focuses on game theoretical aspects by analyzing strategic
behavior of self-interested agents. From the viewpoint of computer
science, computational problems are addressed, such as finding the
optimal allocation in auctions. As this work concentrates on
computational aspects, we assume that the reader has a stronger
background in computer science than in economics. Thus, in this
chapter we will point out the main ideas of the economical
perspective to provide some basic knowledge in this area.

\section{Mechanism Design}
\subsection{Definition}
Mechanism design was introduced by \textcite{Hu60}. It
aims at implementing system-wide solutions to problems in
non-cooperative environments with multiple self-interested agents.
Such problems can be political elections, public projects in which
the participants themselves have to invest money, or allocation
problems. Given that agents hold only private information about
their preferences, a structure has to be chosen in which in
equilibrium each agent behaves according to the designer's or principal's
intentions. The designer can either act on behalf of the society, for
example when collecting taxes for a public project, or she can
pursue self-interests when, for instance, being an auctioneer.

Since the agents' information is private, the principal faces the
problem that the agents might lie about their real valuations in
order to influence the outcome according to their preferences. In
most cases, whenever such manipulations occur, they damage the
resulting system-wide welfare \parencite{NiRo00}. Thus, simply asking
the participants to reveal their preferences is unfavorable.
Therefore, the principal has to define other rules which lead to
the desired outcome. The most common solution to this problem is
to introduce monetary transfers providing incentives for the
agents to behave truthfully.

In mechanism design two economic areas are joined: \emph{game
theory} and \emph{social choice theory}. In game theory the
agents' strategies are analyzed, and in social choice theory an
outcome is selected according to a set of agents' preferences. The
outcome in social choice theory is determined by a social choice
function, which is to be implemented by a mechanism. Formally we
have a set of possible outcomes $O$ and agents $i \in I$, $|I|=n$.
Each agent $i$ has a type $\theta_i \in \Theta_i$ reflecting the
possible preference sequences the agent can have. The type captures
all of the agent's private information relevant to her decision.
The agent's utility $u_i(o,\theta_i)$ over each outcome depends on
her type; while $u_i(o_1,\theta_i)>u_i(o_2,\theta_i)$ means that
the outcome $o_1$ is preferred over the outcome $o_2$. The social
choice function maps from the space of all types $\Theta$ to the
space of all outcomes $O$,
\begin{eqnarray}
% \nonumber to remove numbering (before each equation)
  f:\Theta_1 \times\Theta_2\times...\times\Theta_n\rightarrow O.
\end{eqnarray}
Examples for such social choice functions are allocation problems or political voting
protocols in which a candidate or a party is chosen. The most common objective of a social choice function is
the maximization of the social welfare, the so called
\emph{allocative-efficiency} \parencite{Pa01}. It maximizes the sum of
all utilities over all agents:
\begin{eqnarray}
\label{Efficient}
% \nonumber to remove numbering (before each equation)
  f(\theta)=\argmax_{o \in O} \sum_{i \in I} u_i(o,\theta_i).
\end{eqnarray}
Another objective is \emph{individual rationality}; the agent's
payoff is never less when participating in the mechanism than her
payoff without participating. Additionally there is \emph{Pareto
optimality}. An outcome is Pareto optimal whenever none of the
agents could perform better without causing another agent to
perform worse than in the current situation.

So far, we have learned what a social choice function is, and what
typical objectives for the choices of outcomes are. Now, a
mechanism has to be found which implements a given social choice
function with one or several of these objectives. For this
purpose, the agents' possible strategies have to be specified
together with an outcome function based on these strategies. The
mechanism should guarantee an implementation despite the
self-interest of the agents \parencite{Pa01}. Mathematically, a
mechanism $M$ is defined on the strategy spaces $S_i$ of the
agents:
\begin{equation}\label{MechanismDefinition}
  M=((S_1,...S_n),g(\cdot))\\
  g:S_1\times...\times S_n\rightarrow O,
\end{equation}
where $g$ is an outcome function and $S_i$ denotes all strategies
or actions an agent $i$ is allowed to take. A mechanism implements
a social choice function if there is an equilibrium strategy
profile $s^\ast(\cdot)=(s^\ast_1(\cdot),...,s^\ast_n(\cdot))$ of
the game induced by $M$ so that
\begin{equation}\label{MechanismImplements}
    g(s^\ast_1(\theta_1),...,s^\ast_n(\theta_n))=f(\theta_1,...,\theta_n),
      \quad \forall
      (\theta_1,...,\theta_n)\in(\Theta_1,...,\Theta_n),
\end{equation}
where $s^\ast_i(\theta_i)$ is the strategy agent $i$ with type
$\theta_i$ plays in the equilibrium. Please note that the
equilibrium concept is not specified in this definition. It could,
for example, be a Nash equilibrium. In this case, given the other
players $j,$ $j \neq i$, conform to the equilibrium strategies
$s^\ast_j(\theta_j)$, no other player $i$ has an incentive to
unilaterally deviate from her equilibrium strategy. Other examples
are the dominant strategy or the Bayes-Nash strategy equilibrium.
The dominant strategy equilibrium facilitates it for the agents
since the optimal strategy for an agent is independent of any
strategies the other agents could play. Thus, the agents do not
need to speculate about the way the others might behave. Informally,
we could say that the concept of dominant strategies "removes game
theory from the problem" \textcite[p.~5]{Pa01}. The Bayes-Nash
equilibrium is similar to Nash equilibriums, but assumes that
agents have incomplete information about the opponents' types.
Therefore, agents use probability functions to speculate about the
other agents' preferences \parencite{OsRu94}.

\subsection{Revelation Principle and Gibbard-Satterthwaite Theorem}
In equation \ref{MechanismDefinition}, we see that a mechanism
defines the available strategies and the function for selecting an
outcome. It is necessary that these strategies are kept simple so
that they can be applied by the agents. The easiest strategies
occur when choosing a \emph{direct mechanism} asking the agents to
report their types directly to the principal, $S_i=\Theta_i$.
Direct mechanisms lead to a centralization of the problem as
agents report their types to a center that determines the outcome
and reports it back to the agents. On the contrary, when applying
\emph{indirect mechanisms} agents have to think about how to
transform their type into a strategy and the latter is reported to
the mechanism. In other words,
\begin{quote} "the computations that go on within the mind of any
bidder in the non-direct mechanism are shifted to become part of
the mechanism in the direct mechanism". \textcite[p.~712]{McMc87}
\end{quote}
When applying these direct mechanisms agents may still lie about
their true types. Mechanisms which, in contrast, succeed in
establishing an equilibrium in which all agents tell the truth,
are called \emph{incentive-compatible}. In this case, it is in the
interest of all agents to report their true types,
$s^\ast_i(\theta_i) = \theta_i,~\forall \theta_i \in \Theta_i$.
Further, if telling the truth is a dominant strategy, the
mechanism is called \emph{strategy-proof}. As will be shown later
on, this can be achieved by the \emph{Vickrey-Clarke-Grooves}
(VCG) mechanism.

We learned that the equilibrium strategy profile $s^\ast(\cdot)$
does not determine the concept of equilibrium. Some equilibrium
concept must be chosen and implemented together with the mechanism. In the
worst case, in order to find out if a certain social choice
function can be implemented by a certain mechanism with, for
instance, dominant strategies, one would have to consider all
possible mechanisms. However, research on mechanism design led to
the \emph{revelation principle} as a solution to this. It states
that for any mechanism, there is a direct, incentive-compatible
mechanism with the same outcome \parencite{McMc87}. An intuitive
explanation for this principle consists in: the transformation
from types into strategies, which occurs in the agents' minds in
indirect mechanisms, and which is used as a filter in the direct mechanism.
That is, the direct mechanism first filters all reports of the
agents and simulates the indirect mechanism with the filtered
input. This principle is valid for the optimal mechanism as well.
Thus, the search for a mechanism can focus on direct
mechanisms. Therefore, if no direct mechanism can implement a
given social choice function, then no indirect mechanism will do
so.

In contrast to the positive result of the revelation principle,
there also exists a negative result, the
\emph{Gibbard-Satterthwaite} theorem. According to it, it is
impossible to find a mechanism with certain positive
characteristics. To understand the theorem, first note that a
social choice function is \emph{truthfully implementable} if and
only if the dominant strategy is to reveal the truth. Furthermore,
a social choice function $f$ is \emph{onto} if for each $o \in O$
at least one element in $\Theta$ exists so that $f$ maps to $o$.
Finally, a social choice function $f$ is \emph{dictatorial}
whenever there is a dictator $j$ among the agents so that for all
outcomes, $o_j$ is strictly preferred to another outcome $o_k$
whenever the dictator $j$ strictly prefers $o_j$ to $o_k$.
Obviously, this is an unwanted characteristic. It turns the
Gibbard-Satterthwaite theorem impractical for real-life mechanisms
since they allow manipulation.

\textbf{Gibbard-Satterthwaite Theorem:} \emph{Given $O$ is finite,
$|O|\geq 3$, and the social choice function $f$ is onto, then $f$
is truthfully implementable in dominant strategies if and only if
f is dictatorial}.

According to the theorem it is impossible to elicit the truth if
dominant strategies exist. However, despite this result, the
theorem can be circumvented by placing restrictions on the agents'
preferences, the way it is done in the VCG mechanism.

\subsection{Vickrey-Clarke-Grooves Mechanism}
The VCG mechanism combines the following important virtues by
introducing a special payment scheme. First, it implements social
choice functions in dominant strategies. Thus, agents do not have
to speculate which strategies the other agents might play, and
they do not need to waste resources on learning about their
competitors' strategies. Second, the mechanism does not have to
make any assumptions about the information agents have on each
other. And, third, the VCG mechanism is allocative-efficient (see
equation \ref{Efficient}), strategy-proof and non-dictatorial.

AND SO ON... 