%!TEX root = ../master.tex


\section{Introduction} \label{chap:introduction}
Scientific papers have standardized structures. There are different standards for different types of papers. You should make sure that the paragraphs are not too short (i.e., at least 3-4 sentences). This is still a rather short paragraph.

\noident This would then be a new paragraph. But this paragraph would be too short.


\section{Brief Overview of the Content of Chapters}
Below we give you a few tips for writing the different chapters.


\subsection{Introduction}
\begin{enumerate}
    \item Problem definition and motivation
    \item State of research, building on this, identify research gap and research question(s)
    \item Aim of the work and own methodological approach to answering the research question(s): Paragraph usually begins with: The goal of this thesis/work/manuscript is ...
    \item Expected scientific (and practical) contribution [=Contribution]
    \item Sometimes follows: Outline [formulated]
\end{enumerate}


\subsection{Chapter on Related Work/Theories/Literature}
\begin{enumerate}
    \item What theories and principles are there?
    \item Focus on a few, central ones
    \item Absolute basics do not need to be explained. You can assume that the readers already know a lot. (specialist audience!)
    \item Often subdivided into subheadings according to the different types of foundations/theories
    \item Not yet refer to own work and own approach, so never "pre-reference"
\end{enumerate}

The aim is to make the research gap clear. You are also welcome to explicitly name and elaborate on this. [This is the only place where implicit reference is made to your own work.]

The other chapters between the basic chapters and the conclusion are individually very different. Your own contribution should be in the foreground in these further chapters.


\subsection{Discussion}
\begin{enumerate}
    \item The discussion reflects the results of the work
    \begin{enumerate}
        \item In empirical works, the results section reports the data situation (e.g., hypothesis tests) in a very dry manner and these results are only interpreted in the discussion.
        \item In purely normative papers, the discussion often coincides with the presentation of the results, as the two parts are more difficult to separate.
    \end{enumerate}
\end{enumerate}


\subsection{Conclusion}
Depending on the length of the thesis, this chapter does not need to be subdivided.
\begin{enumerate}
    \item Summary
    \begin{enumerate}
        \item Varies depending on the level of detail of the previous discussion. In empirical papers: Brief summary of findings and higher-level analysis of results
        \item Reference to research questions and the extent to which they have been answered
    \end{enumerate}
    \item Limitations/critical appraisal and outlook
    \begin{enumerate}
        \item What problems were encountered in the present work?
        \item How can these be addressed in future work?
        \item What other future, related research topics should be addressed in the future?
    \end{enumerate}
    \item Contribution:
    \begin{enumerate}
        \item What conclusions can be drawn?
        \item What contributions does the work make to research and practice? [Contribution]
    \end{enumerate}
\end{enumerate}





