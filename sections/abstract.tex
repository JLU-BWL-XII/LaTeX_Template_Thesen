%!TEX root = ../master.tex
\section*{Abstract}
[Hier ein Beispiel für ein Abstract. Ihr Abstract sollte 150-250 Wörter haben] We investigate how each of the two steps that are typically supported by purchasing platforms ― filtering and joint evaluation ― affects the success of a prosocial microlending platform. Users of such platforms lend money interest-free to people in need, such as small-scale entrepreneurs from developing countries. We hypothesize that while attribute-based filtering can reduce the decision effort and provide guidance, which is often perceived as helpful in purchasing decisions, it may be perceived as inappropriate and restrictive in the prosocial microlending domain, thereby reducing users’ choice satisfaction. Building on evaluability theory, we further hypothesize that joint evaluation is a double-edged sword: Jointly evaluating more than one alternative increases choice satisfaction by facilitating evaluability, as alternatives can serve as reference points, and because not being able to compare alternatives could feel restrictive. However, jointly evaluating alternatives also highlights conflicts and tradeoffs between alternatives and thereby decreases users’ willingness-to-contribute to the alternative they finally choose. We test our hypotheses in an incentivized lab experiment, using real prosocial lending decisions. Our findings suggest that offering attribute-based filters does not increase a platform’s success, and confirm that joint evaluation is a double-edged sword. Platforms have to trade off decreased choice satisfaction with increased willingness-to-contribute.

\newpage
