% ------------------------------------------------------------------------
% Template zur Erstellung von Abschlussarbeiten an der Professur für
% Digitalisierung, E-Business und Operations Management
% Erstellerin: Jella Pfeiffer
% Datum 08.10.2019
% Überarbeitet von Pascal Heßler am 13.08.2020
% ------------------------------------------------------------------------

% ------------------------------------------------------------------------
% Allgemeine Einstellungen
% ------------------------------------------------------------------------
\documentclass[
    12pt,
    paper=a4,
    headings=small,               % kleinere Überschriften
    twoside=false,                % einseitig, nur rechte Seiten
    listof=totoc,                 % Listen im Inhaltsverzeichnis aufnehmen
    bibliography=totoc,           % Literaturverzeichnis ins Inhltsvz. aufnehmen
    headsepline                   % Trennlinie unter Kopfzeile
    ]{scrbook}

\usepackage[paper=a4paper,left=30mm,right=30mm,top=40mm,bottom=40mm]{geometry}
\usepackage[utf8]{inputenc}       % Damit können Umlaute ganz normal geschrieben werden.
% \usepackage[english]{babel}     % Verwende deutsche, bzw. amerikanische Silbentrennung
\usepackage[ngerman]{babel}       % Deutsche AlternativeMcMc87
\usepackage{scrlayer-scrpage}     % Paket für Kopf und Fußzeile. Kommentar: "scrpage2" veraltet in "scrlayer-scrpage" geändert
\pagestyle{scrheadings}           % Kopzeilenseitenstil
\usepackage{setspace}             % Zeilenabstand imlpementiert den Befehl 
% ------------------------------------------------------------------------
% Literatur
% ------------------------------------------------------------------------
% Achtung zur Einbindung der Literatur wird hier Biber verwendet und nich der Standard BibTex. In Sublime bracu man nichts verändern.
% 1. Sicherstellen, dass Biber installiert ist ->  in Tex Live vorinstalliert
% 2. Stellt eure Schreibumgebung um -> In TeXstudio: Optionen -> TeXstudio konfigurieren -> Erzeugen -> Standard Bibliographieprogramm
\usepackage[backend=biber,style=apa,natbib=true]{biblatex}
\DeclareLanguageMapping{american}{american-apa}         % Sprach Einstellung
%\DeclareLanguageMapping{ngerman}{ngerman-apa}          % Deutsche Alternative
% \addbibresource{BibliographieAbschlussarbeit.bib}       % Legt die Datei fest in der die Literatur liegt (z. B.: Export aus Citavi)
\bibliography{BibliographieAbschlussarbeit.bib}
\usepackage{breakcites}                                 % Falls Zitationen nicht ans Zeilenende passen
\usepackage{csquotes}                                   % Erweitert die Möglichkeiten beim zitieren
% ------------------------------------------------------------------------
% Mathematische Symbole
% ------------------------------------------------------------------------
\usepackage{amssymb}
\usepackage{amsmath}        % Formattierung von Tabellen und Matritzen
\usepackage{amsfonts}

% ------------------------------------------------------------------------
% Grafiken
% ------------------------------------------------------------------------
\usepackage{graphicx}       % Zum Einbinden von Grafiken
\usepackage{subfigure}      % Um Bilder innerhalb einer Figure anzuordnen (Bild a b c d...)

% ------------------------------------------------------------------------
% Tabellen
% Um einfache Tabellen zu erstellen kann https://www.tablesgenerator.com/ verwendet werden.
% ------------------------------------------------------------------------
\usepackage{tabularx}       % Ermöglicht weitere Tabellen Einstellungen
\usepackage{multirow}       % Ermöglicht Zellen zu verbinden
\usepackage{booktabs}       % Tabellen Midline Topline etc.
\usepackage{float}          % Tabelen Positionierung
\usepackage{longtable}      % Ermöglicht Tabellen über mehrere Seiten hinweg, so dass sie noch dasselbe Format besitzten ftp://ftp.fu-berlin.de/tex/CTAN/macros/latex/required/tools/longtable.pdf
% **********************************************************************************************************************
% Folgende Pakete sind rein optional und etwas komplexer in ihrer Umsetzung nur für etwas erfahrenere LaTeX User
% **********************************************************************************************************************
% \usepackage{siunitx}      % Ausrichten von Tabellen an dezimalstellen (Umsetzung etwas komplizierter)
% \usepackage{rccol}        % Ein Neues Format R (wird wahrscheinlich mit dem Folgenden R zu Problemen führen) und sorgt, dass nach , ausgerichtet wird und automatisches runden in Tabellen wird ermöglicht
% \usepackage{fltpoint}     % Gehört zu rccol
% \usepackage{dcolumn}      % Gehört zu rccol
%
% \newcolumntype{L}[1]{>{\raggedright\arraybackslash}p{#1}}       % Linksbündig mit Breitenangabe
% \newcolumntype{C}[1]{>{\centering\arraybackslash}p{#1}}         % Zentriert mit Breitenangabe
% \newcolumntype{R}[1]{>{\raggedleft\arraybackslash}p{#1}}        % Rechtsbündig mit Breitenangabe
% \newcolumntype{x}[1]{!{\centering\arraybackslash\vrule width #1}}

% ------------------------------------------------------------------------
% Sonstige
% ------------------------------------------------------------------------
\usepackage{scrhack}        % Löst unnötige Warnungen!
\usepackage{color}          % Ermöglicht Text einzufärben \pagecolor{FARBE}  \color{FARBE} \textcolor{FARBE}{TEXT} \colorbox{FARBE}{TEXT}
\usepackage{hyperref}       % Ermöglicht URLS schreibweise ist z.B. \url{http://www.uni-giessen.de}

\usepackage{eurosym}        % Für Euro Symbol \euro{} oder \eruo (Unterschieden sich bezüglich Leerzeichen danach)

\usepackage{framed}         % Einrahmen von Texten mit der shaded-Umgebung.

\usepackage{pdflscape}      % Querformat möglich über \begin{landscape} \end{landscape}
% ------------------------------------------------------------------------
% Fein Justierungen
% ------------------------------------------------------------------------
\DeclareMathOperator*{\argmax}{arg\,max}

\renewcommand{\floatpagefraction}{.9}     % vorher: .5
\newenvironment{packed_enum}{
\begin{enumerate}
  \setlength{\itemsep}{1pt}
  \setlength{\parskip}{0pt}
  \setlength{\parsep}{0pt}
}{\end{enumerate}}

%\setlength{\parindent}{0pt}                    % Kein Einzug beim Absatzbegin
\setlength{\parskip}{1pt plus 0pt minus 1pt}    % Abstand zwischen 2 Absätzen


% Kapitel Überschriften in Schriftart mit Serifen
\setkomafont{sectioning}{\normalfont\normalcolor\bfseries}
% Gestaltung der Kopfzeilen
\ohead{\pagemark} \cfoot{} \cohead{} \ihead{\headmark}
\setkomafont{pagehead}{\normalfont\bfseries}
%\setkomafont{pagenumber}{\normalfont\bfseries} \automark{section}

\hypersetup{hidelinks}                 % Versteckt die Links im PDF (optisch)
% ----- ende der präambel ----------------------------------
% Start des Dokuments
% ----------------------------------------------------------
\begin{document} 
\frontmatter                           % Corspann, Kapitel römisch nummeriert

%!TEX root = ../master.tex
% Die Titelseite der Arbeit

\begin{titlepage}

\begin{center} % zentrieren

  % Logo der Uni Giessen bitte suchen und einfuegen
  \begin{figure}[ht]
    \centering
    \includegraphics{graphics/logo.png}
  \end{figure}

  % Vertikaler Zwischenraum
  \bigskip
  \vfill
  % Titel der Arbeit und Typ der Arbeit, umrandet
    \begin{framed}
    \begin{center}
      \textsc{{\Large Titel\\
      Untertitel\\}}
                                % Letztes \\ ist wichtig, beginnt eine neue Zeile f{\"u}r die Art der Arbeit

      \bigskip

                                % Art der Arbeit, ggf. auszutauschen gegen Seminar- oder Bachelorarbeit
      \textbf{Masterthesis}
    \end{center}
    \end{framed}
    \vfill
    \vfill

%@Lukas Münzel: bitte austauschen durch eingereicht am, Name, Vorname, Matrikelnummer, Studiengang, private Adresse, Telefonnummer und E-Mail Adresse

  % Daten des Erstellers, Einreichungsdatum
  % in einer Tabelle ausgerichtet
  \begin{tabular*}{0.62\textwidth}{r@{\extracolsep{\fill}}l}
    eingereicht im: & Oktober 2019\\\\
    von: & XXX\\
    & geboren am XX. Oktober 2010\\
    \\
    Matrikelnummer: & XXX\\
    Studiengang: & XXX\\
    Private Adresse: & XXX\\
    Telefonnummer: & XXX\\
    E-Mail-Adresse: & XXX
  \end{tabular*}
  \vfill
  \vfill


  \rule{\textwidth}{.4pt}\\ % vertikale Linie
  Justus-Liebig-Universität Gießen\\
  Professur für Digitalisierung, E-Business und Operations Management\\
  Licherstraße 74 \\
  35394  Gießen\\
  \small Internet: \url{https://www.uni-giessen.de/fbz/fb02/fb/professuren/bwl/e-business-operations-management}
\end{center}

\end{titlepage} % Ende des Titelblatts

%%% Local Variables:
%%% mode: latex
%%% TeX-master: "~/Documents/DA-Vorlage/beispiel/da-beispiel"
%%% End:
               % Titelseite einbinden
\tableofcontents                       % Fügt Inhaltsverzeichnis ein
% %!TEX root = ../master.tex
%-------------------------------------
%
%    Stichwortverzeichnis (ca. 5-10 Stichworte, welche den Inhalt der Arbeit beschreiben
%
%-------------------------------------

\chapter{Keywords} % beachte addchap
\begin{labeling}{1234567890}
        \item Duplicate Elimination
        \item Genetic Algorithm
        \item Greedy Heuristic
        \item Integer Program
        \item Meta-Heuristic
        \item Multi-Dimensional Knapsack Problem
        \item Multi-Unit Combinatorial Auction
        \item Phenotypic Distance
        \item Winner Determination Problem
\end{labeling}
            % Fügt Stichwortverzeichnis ein    (nur wenn nötig)
% \listoffigures                         % Fügt Abbildungsverzeichnis ein   (nur wenn nötig)
% \listoftables                          % Fügt Tabellenverzeichnis ein     (nur wenn nötig)
%-------------------------------------
%
%    minimales abk�rzungsverzeichnis
%
%-------------------------------------

\addchap{Abbreviations} % beachte addchap
\begin{labeling}{1234567890}
        \item[CA] combinatorial auction
        \item[VCG] Vickrey-Clarke-Grooves
\end{labeling}
       % Beispiel eines handerstellten Verzeichnisses Kommentar: Abkürzungsverzeichnisse können auch über \usepackage[acronyms, shortcuts, translate=babel, nomain]{glossaries} generiert werden.

% ------------------------------------------------------------------------
% Hauptarbeit (hier werden alle Kapitel eingebunden)
% ------------------------------------------------------------------------
\mainmatter                            % Hauptteil, Kapitel lateinisch nummeriert
\onehalfspacing
% Theoretisch kann man auch alles in einem Dokument haben. Das ist jedoch unübersichtlich. Daher werden hier die einzelnen Kapitel eingebunden.
%!TEX root = ../master.tex
\chapter{Fromeln}
Unter Verwenung der schnell schreibweise mittels eines einzelnen \emph{\$}-Zeichens kann relativ simpel eine Formel im Text verwendet werden.
$ c(\tau) =\frac{1}{|C|(|C|-1)} \sum_{j \in C\backslash \{i\}}1_{d_{ij}>\tau}$\\\\
Unter Verwenung der schnell schreibweise mittels zwei \emph{\$\$}-Zeichen wird die Formel in einem eigenen Absatz angezeigt:

$$ c(\tau) =\frac{1}{|C|(|C|-1)} \sum_{j \in C\backslash \{i\}}1_{d_{ij}>\tau}$$

Unter Verwenung der Umgebung \emph{align} wird die Formel automatisch Nummeriert.
\begin{align}
    c(\tau) =\frac{1}{|C|(|C|-1)} \sum_{j \in C\backslash \{i\}}1_{d_{ij}>\tau}
\end{align}

$$X_i^2$$
%!TEX root = ../master.tex
\chapter{Tabellen}
Tabellen in LaTeX sind relativ umständlich händisch zu erstellen. Zum Glück gibt es aber Alternativen, welche einem deutlich erleichtern schöne Tabellen zu erstellen.
\begin{itemize}
    \item \url{https://www.tablesgenerator.com/#}
    \item Plugins für Excel (Excel2Latex)
    \item Stata Pakete (estout)
\end{itemize}

Beispiel für eine einfache Tabelle:
\begin{table}[h]
\centering
\caption{Ergebnisse}
\label{tab:easy_table}
\begin{tabular}{l|lll|lll}
\hline\hline \multicolumn{4}{l}{} \\ [-10pt]
Datensatz & \multicolumn{3}{c|}{Trainingsdatensatz} & \multicolumn{3}{c}{Testsdatensatz} \\
Modelle &                \multicolumn{1}{c}{ $N$} &     RMSE &      MAE &           \multicolumn{1}{c}{ $N$}&      RMSE &      MAE \\
\midrule
Fix-Effekt              &           2.040.264 & 66,189     &     27,822 &    679.821 & 67,029 & 27,877 \\
Gradient Boost    &           2.210.146 & 5,9302     &     3,3176 &    736.716 & 10,442 &  3,8284 \\
\hline\hline 
\end{tabular}

\end{table}

Ein Vorteil von Latex ist, dass man mit dem Befehl \emph{ \textbackslash ref\{Label\} } auf jede Tabelle verweisen kann. Ein solcher Verweis ruft automatisch die aktuelle Nummer der Tabelle auf \ref{tab:easy_table} und verlinkt diese mit der Tabelle.\\
Es kann Sinn machen Tabellen in getrennten Dateien abzulegen. Somit hat man die Möglichkeit diese durch ein Programm, zum Beispiel Stata, direkt zu überschreiben und die aktuellste Tabelle wird direkt in euer LaTeX Dokument eingebunden.
%!TEX root = ../master.tex
\chapter{Abbildungen}
Man kann viele verschiedene Dateiformate in LaTeX einbinden, die beste Variante sind jedoch Vektorgrafiken. Diese sind beliebig skalierbar und sind somit immer gut zu lesen bzw. nie verpixelt. Wenn man Pixel-Grafiken verwendet sollte immer darauf geachtet werden, dass eine möglichst hohe Auflösung der Datei sichergestellt wird. 
 \begin{figure}[H]
    \centering
    \includegraphics[width=0.8\linewidth]{graphics/summary.png}
    \caption{Abbildung der SHAP-Werte für die 15 wichtigsten Variablen}
    \label{fig:SHAP}
\end{figure}  
Um Bilder in LaTeX einzubinden wird die Umgebung \emph{figure} verwendet. In dieser wird über den Befehl \emph{\textbackslash includegraphics[Optionen]\{Dateipfad\}} die Datei aufgerufen. Eine besonders interessante Option ist \emph{width=0.8\textbackslash linewidth} welche dafür sorgt, dass die Abbildung immer genau gleich groß ist, nämlich genau 80\% der Textweite. \\
Auch auf Abbildungen kann verwiesen werden (Abbildung \ref{fig:SHAP}).
%!TEX root = ../master.tex
\chapter{Zitieren}
Wenn sie im Text zitieren wollen verwenden sie \parencite{NiRo00}  \\ \textcite{Hu60} \\ \textcite[97]{McMc87} \textcite[7374]{Hu60}
% ------------------------------------------------------------------------
% Hauptarbeit ENDE
% ------------------------------------------------------------------------

% ------------------------------------------------------------------------
% Erstellen des Literaturverzeichnisses
% ------------------------------------------------------------------------
\printbibliography[heading=bibintoc]  
% ------------------------------------------------------------------------
% Einbinden des Anhangs
% ------------------------------------------------------------------------
\appendix                               % Kapitel werden ab hier Alphabetisch aufgenommen
\chapter{Appendix}\label{Appendix}




\begin{table}
\begin{center}
\caption{Size of the search space for BASIC strategies ($n=4$).}
\label{tab:sizeSearchSpace}
 \begin{tabular}{r|r|r}
      m&s&search space size\\
      \hline
      4&1&64\\
      &2&4096\\
      &3&262,144\\
      &4&16,777,216\\
      &5&1,073,741,824\\
      7&1&262144\\
      &2&16777216\\
      &3&68,719,476,736\\
      &4&2.81E+14\\
      &5&1.15E+18\\
\end{tabular}
\end{center}
\end{table}



% ------------------------------------------------------------------------
% Einbinden der Eidesstaatlichen Erklärung
% ------------------------------------------------------------------------
\backmatter                             % Ab hier wird nichts mehr ins Inhaltsverzeichnis aufgenommen
\chapter*{Eidesstattliche Erkl\"{a}rung}
\thispagestyle{empty}
Ich versichere, dass ich meine Diplomarbeit ohne Hilfe Dritter und ohne Benutzung
anderer als der angegebenen Quellen und Hilfsmittel angefertigt und die den benutzten
Quellen w\"{o}rtlich oder inhaltlich entnommenen Stellen als solche kenntlich gemacht habe.
Diese Arbeit hat in gleicher oder \"{a}hnlicher Form noch keiner Pr\"{u}fungsbeh\"{o}rde
vorgelegen.
\bigskip

\raggedright{Mannheim, den 26.10.2006} \bigskip \bigskip \bigskip

Jella Pfeiffer

\end{document}
% ----------------------------------------------------------
% Ende
% ----------------------------------------------------------
